\begin{abstract}
In recent years, there has been a growth of interest for geo-distributed applications, and for protocols and algorithms that can be used for such systems. The services that use these systems often have need to store their data in some structured way, to improve the performance of the system or to have a more convenient type of storage for their specific needs. For this reason, there is a need for distributed data structures that are both efficient and usable in geographically distributed environments.

One possible technique to obtain a distributed system on multiple data centers across the globe is through State Machine Replication, which allows us to have a consistent and deterministic state across all the replicas. This has an impact on performance, though, particularly on the latency of the communications. A way to solve this is by partitioning, or sharding, the state among the different distributed replicas[should I mention GeoPaxos here]. Partitioning becomes especially complicated when the state that is to be partitioned is a complicated data structure. The goal would be to have a partitioning that is fast to compute, deterministic, and that it improves the performance of the distributed system.

In this report we analyze the problem of finding an optimal partitoning algorithm for a distributed B+ Tree to achieve minimum average latency of communication across the network of replicas, particularly focusing on the time performance of the algorithm and on the quality of the final partitioning. We go over different approaches, comparing the quality of the repartitioning algorithms, and we then test them on GeoPaxos, a capable Paxos-based algorithm for geo-distributed systems that allows us to build and manage a B+ Tree on top of it.
\end{abstract}