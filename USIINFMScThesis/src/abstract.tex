\begin{abstract}
In recent years, there has been a growth of interest for geo-distributed applications, and for protocols and algorithms that can be used for such systems. The services that use these systems often need to store their data in some structured way, to improve the performance of the system or to have a more convenient type of storage for their specific needs. For this reason, there is a need for distributed data structures that are both efficient and usable in geographically distributed environments.

One possible technique to obtain a distributed system on multiple data centers across the globe is through State Machine Replication, which allows us to have a consistent and deterministic state across all the replicas. This has an impact on performance, though, particularly on the latency of the communications. A common way to solve this is by partitioning, or sharding, the state among the different distributed replicas. An alternative solution is to fully replicate the data structure, and instead partition the responsibility of ordering the operations, as it is done by GeoPaxos \citep{geopaxos}. This approach can become especially challenging when the data structure used is complex, like a B+tree: since the data structure is fully replicated, all operations have to be deterministic and executed in the right order, to ensure that the state is consistent among the different replicas.

In this master thesis we analyze the problem of finding efficient algorithms for B+trees to partition the responsibility of ordering operations. The goal  is to minimize the average latency of communication across the network of replicas, while also focusing on the time performance of the algorithm and on the quality of the final partitioning. In this thesis we evaluate different approaches, comparing the quality and efficiency of the repartitioning algorithms, and we then test them on GeoPaxos, a Paxos-based algorithm for geo-distributed systems that allows us to build and manage a B+tree on top of it.
\end{abstract}