\chapter{Introduction}\label{sec:introduction}
In recent years, the number of distributed systems and that are been used privately by companies or that are offered as services to the public has been growing quickly. This brought the need for new and better ways to tackle the problems that distributed systems have to face, such as availability, consistency and latency. GeoPaxos is a particularly valid solution for geo-distributed systems, but it works on a per-object granularity. Building an efficient distributed data structure on top of GeoPaxos would bring the performance of GeoPaxos closer to the needs of more distributed applications.


\section{Motivation}\label{sec:motivation}
For a distributed system we want to have availability, consistency and low latency, but usually one comes at the cost of the others. Having multiple replicas of the system is a solution for availability, but this still comes with high latency for the clients; placing the replicas close to the clients does not solve the issue, as the replicas will still have to communicate to ensure a consistent state, therefore still with high latency. If the system shows locality, meaning that different replicas will face requests on different objects, then a solution for this is to shard or partition the state, so that different replicas only handle part of the state. This allows to place the replicas that handle specific partitions close to the clients that want to access those partitions the most.

A distributed structure such as a B+ Tree would be particularly suitable. The problem with this is that, as the state is partitioned, if a client performs operation on multiple partitions, the replicas will have to communicate to order the requests, while still ensuring the consistency of the state. Also, if the locality of the operations changes over time, there will be the need to repartition the state, which involves the data of the whole state. We therefore have the need to provide a way to perform a repartition on a distributed B+ Tree that is deterministic and as close to optimal and efficient as possible.

\section{Main Objective}\label{sec:main-objective}
A fully-replicated distributed B+ Tree implemented on top of GeoPaxos needs to have each of its nodes assigned to one or more partition, or group, so that we know which partitions will have to communicate and synchronize to order the operations for each node. Finding the right partitions depends on the history of operations on each node received by the different groups.

The process to find the right partitions is called repartition. We want to do the repartition often, since we cannot expect the workload of the system to remain constant over time. The complexity of the repartition problem grows linearly with the number of objects and exponentially with the number of groups; this means that given a tree big enough, the repartition can take an important amount of time, in particular if performed frequently. Therefore the goal of this report is to come up with one or more algorithms to perform the repartition in an efficient way, so that we can be free to use a distributed structure as a B+ Tree on top of GeoPaxos, with all its benefits, without adding notable performance costs.

\section{The Structure of this Report}\label{the-structure-of-this-report}
The remaining chapters of this report will be structured as follows.

First there will be a brief overview of the background needed to better understand the core part of the report; the background will go over some of the basics of distributed systems and algorithms, including terminology and fundamental concepts of this branch. The background will then continue with the introduction to the problem of Consensus, followed by the Paxos algorithm, which is used to solve the consensus problem, and finally GeoPaxos[cite], an improved version of Paxos particularly suited for geo-distributed systems. 

Then we will discuss the usage of a B+ Tree with GeoPaxos, including the advantages and the challenges that come with it. Finally we will discuss the repartition problem of a distributed B+ Tree, which is the core subject of this report. This part will include an introduction to the problem, followed by a list of attempted solutions for improving the repartition process. The chapter will then be concluded with a various tests, both on the performance of the solutions presented, and on the execution of GeoPaxos with the different repartition algorithms.

The last chapter will be about drawing the conclusions of this work, discussing possible improvements, potential future work, and some final remarks.